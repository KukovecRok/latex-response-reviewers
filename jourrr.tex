\documentclass{jourrr}

% Path to your signature (if you have one). 
\signaturePath{imgs/jourrr-signature.png}
%----------------------------------------------------------------------
% Your Data
%----------------------------------------------------------------------
% change according to your data
\NameSurname{Name Surname}
%----------------------------------------------------------------------
% Article data
%----------------------------------------------------------------------
% journal name
\JournalName{JournalName}
% title of your paper
\Title{A review of journal name title}
% name of the EiC
\Editor{Prof. Name Surname}
% TODO make this optional
\SpecialIssue{SI 2022}
% Manuscript ID
\ManuscriptID{1234567890}

%----------------------------------------------------------------------
% Response letter main part
%----------------------------------------------------------------------
% No need to change the following part - here for customization
% Line spacing
\renewcommand{\baselinestretch}{1.3}
% Addressing the Person
\PersonAddressing{Dear Editor}

% change valediction in \valediction{ \OPTION_HERE } or \valediction{ Custom text }
% \yoursfaithfully % In case we do not know the name of the Editor
% \yourssincerely % In case we know the name of the Editor
% \yourstruly % American version of Yours faithfully
% \bestregards
\valediction{\yourssincerely}

\signature{\pNameSurname (On behalf of all Authors)\newline \break
University of City}

%----------------------------------------------------------------------

\Introduction{We would like to Thank you very much for your help, kindness, and your willingness by giving us a chance to submit our manuscript to the review process of your journal. We have studied each comment carefully and had acted upon every point raised by the reviewers. \newline \break
We also revised our manuscript as suggested by reviewers. We made changes which enhance the written quality and flow of the paper. \newline \break
Below, we have highlighted responses to each comment from the reviewers and indicated possible solutions to them. \newline \break
Looking forward to hearing from you soon.}

\ReviewerThank{Authors would like to sincerely thank the both Reviewers for the time spent on reading and commenting the manuscript. The given remarks are very thorough and extremely helpful to overcome the shortcomings and ambiguities in the manuscript. The Reviewer's effort is highly appreciated. In what follows, a point-by-point reply to the Reviewer’s remarks, comments and suggestions is given.}

\begin{document}
\loadgeometry{title}
\begin{titlepage}

\large % Font size
\responseheader{
    Responses to Reviewers Comments on\\
    ``\pJournalName``\\
    (\pManuscriptID)
    }
%----------------------------------------------------------------------

\begin{center}
Responses to Reviewers Comments on\\
    ``\pJournalName``\\
    (\pManuscriptID)
\end{center}

% Today's date
\today

\vspace*{1\baselineskip}
\pPersonAddressing,

\pIntroduction

\vfill

\pvalediction\par

\psignature

% Next line shows your signature. Comment it out if you don't have one / plan to use digital on .pdf
\showSignature{\pSignaturePath}

\end{titlepage}

\pagestyle{empty} 
% Load predefined geometry for document
\loadgeometry{doc}
% Set linespread for this part
\linespread{1.5}\selectfont

\pReviewerThank

\referee{}\newline
Comments to the Author.

General comment of reviewer to the author.

\response{true}{Authors thanks the Reviewer 1 for his/her positive mention about our work.}

\comment{The weights (eq 19) of the objectives will be crucial to the quality of maps
produced, and I guess will be highly problem-specific. how might we go about tuning them?}

\response{false}{In order to satisfy this and also the demands of the other reviewers, the fitness
function was defined as [in pag. 6, para. 3.1.4]:}

\comment{The maximum line length, as the authors note, is also highly problem specific. There might not be a clear cutoff between important and unimportant features where this threshold can be set. Have the authors any thoughts on this? It would be worth discussing this somewhere.}

\response{false}{Authors thanks the Reviewer 1 for his/her comment. The parameter maximum line length is crucial for optimizing algorithm and therefore the new subsection was added to solve the issue [in pag. 9, para. 4.3.1.2]:}

\referee{}\newline
Comments to the Author.

General comment of reviewer to the author.

\response{true}{Authors thanks the Reviewer 1 for his/her positive mention about our work.}

\comment{The weights (eq 19) of the objectives will be crucial to the quality of maps
produced, and I guess will be highly problem-specific. how might we go about tuning them?}

\response{false}{In order to satisfy this and also the demands of the other reviewers, the fitness
function was defined as [in pag. 6, para. 3.1.4]:}

\comment{The maximum line length, as the authors note, is also highly problem specific.
There might not be a clear cutoff between important and unimportant features where this
threshold can be set. Have the authors any thoughts on this? It would be worth discussing
this somewhere.}

\response{false}{Authors thanks the Reviewer 1 for his/her comment. The parameter maximum
line length is crucial for optimizing algorithm and therefore the new subsection was added to
solve the issue [in pag. 9, para. 4.3.1.2]:}

\referee{}\newline
...

\vfill

\pvalediction\\
\pNameSurname

\end{document}